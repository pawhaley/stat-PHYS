\documentclass[12pt]{article}

\usepackage{amssymb,amsmath,amsthm}
\usepackage[top=1in, bottom=1in, left=1.25in, right=1.25in]{geometry}
\usepackage{fancyhdr}
\usepackage{enumerate}
\usepackage[bw,framed,numbered]{mcode}
\usepackage{graphicx}

% Comment the following line to use TeX's default font of Computer Modern.
\usepackage{times,txfonts}

\newtheoremstyle{homework}% name of the style to be used
  {18pt}% measure of space to leave above the theorem. E.g.: 3pt
  {12pt}% measure of space to leave below the theorem. E.g.: 3pt
  {}% name of font to use in the body of the theorem
  {}% measure of space to indent
  {\bfseries}% name of head font
  {:}% punctuation between head and body
  {2ex}% space after theorem head; " " = normal interword space
  {}% Manually specify head
\theoremstyle{homework} 

% Set up an Exercise environment and a Solution label.
\newtheorem*{exercisecore}{Exercise \@currentlabel}
\newenvironment{exercise}[1]
{\def\@currentlabel{#1}\exercisecore}
{\endexercisecore}

\newcommand{\localhead}[1]{\par\smallskip\noindent\textbf{#1}\nobreak\\}%
\newcommand\solution{\localhead{Solution:}}

%%%%%%%%%%%%%%%%%%%%%%%%%%%%%%%%%%%%%%%%%%%%%%%%%%%%%%%%%%%%%%%%%%%%%%%%
%
% Stuff for getting the name/document date/title across the header
\makeatletter
\RequirePackage{fancyhdr}
\pagestyle{fancy}
\fancyfoot[C]{\ifnum \value{page} > 1\relax\thepage\fi}
\fancyhead[L]{\ifx\@doclabel\@empty\else\@doclabel\fi}
\fancyhead[C]{\ifx\@docdate\@empty\else\@docdate\fi}
\fancyhead[R]{\ifx\@docauthor\@empty\else\@docauthor\fi}
\headheight 15pt

\def\doclabel#1{\gdef\@doclabel{#1}}
\doclabel{Use {\tt\textbackslash doclabel\{MY LABEL\}}.}
\def\docdate#1{\gdef\@docdate{#1}}
\docdate{Use {\tt\textbackslash docdate\{MY DATE\}}.}
\def\docauthor#1{\gdef\@docauthor{#1}}
\docauthor{Use {\tt\textbackslash docauthor\{MY NAME\}}.}
\makeatother

% Shortcuts for blackboard bold number sets (reals, integers, etc.)
\newcommand{\Reals}{\ensuremath{\mathbb R}}
\newcommand{\Nats}{\ensuremath{\mathbb N}}
\newcommand{\Ints}{\ensuremath{\mathbb Z}}
\newcommand{\Rats}{\ensuremath{\mathbb Q}}
\newcommand{\Cplx}{\ensuremath{\mathbb C}}
%% Some equivalents that some people may prefer.
\let\RR\Reals
\let\NN\Nats
\let\II\Ints
\let\CC\Cplx

%%%%%%%%%%%%%%%%%%%%%%%%%%%%%%%%%%%%%%%%%%%%%%%%%%%%%%%%%%%%%%%%%%%%%%%%%%%%%%%%%%%%%%%
%%%%%%%%%%%%%%%%%%%%%%%%%%%%%%%%%%%%%%%%%%%%%%%%%%%%%%%%%%%%%%%%%%%%%%%%%%%%%%%%%%%%%%%
% 
% The main document start here.

% The following commands set up the material that appears in the header.

%%%%%%%%%%%%%%%%%%%%%%%%%%%%%%%%%%%%%%%%%%%%%%%%%%%%%%%%%%%%%%%%%%%%%%%%%%%%%%%%%%%%%%%
%%%%%%%%%%%%%%%%%%%%%%%%%%%%%%%%%%%%%%%%%%%%%%%%%%%%%%%%%%%%%%%%%%%%%%%%%%%%%%%%%%%%%%%
% 
% The main document start here.

% The following commands set up the material that appears in the header.
\doclabel{STAT PHYS: Homework 1}
\docauthor{Parker Whaley}
\docdate{Feb 8, 2017}

\newcommand{\vv}{\mathbf{v}}
\begin{document}
\begin{exercise}{1.18}
Recall $\frac{m<v^2>}{2}=\frac{3}{2}k_bT$.  If we fill in we get $\frac{(2*2.3258671 \times 10^{-26} kg)<v^2>}{2}=\frac{3}{2}(1.38064852 \times 10^{-23} m^2 kg s^{-2} K^{-1})(295K)$ we then see $<v^2>=1.3134\times 10^5 m^2/s^2$ or $\text{rms}=362m/s$.
\end{exercise}

\begin{exercise}{1.21}
After a quick bit of research (watching YouTube videos) I have concluded that hailstones colliding with a object is a inelastic process, they bounce back with trivial velocity if they don't stick.  The force exerted by the hailstones would be $F_1=.002kg*\sqrt{2}/2*15m/s*30/s=0.63640N$ thus the pressure would be $1.2728Pa$, whereas the atmospheric pressure is $101,000Pa$.  However before one concludes that a glass window can easily support a hailstorm note that the pressure exerted by the atmosphere is exerted on both sided of the glass and thus does not actually put stress on the glass.
\end{exercise}

\begin{exercise}{1.34}
\begin{enumerate}[(a)]
\item
For $A$ and $C$ there is no change in volume and thus no work is done.  Define $\Delta P=P_2-P_1$ and $\Delta V=V_2-V_1$.  The work done on the system for $D$ is $w_D=P_1 \Delta V$ and for $B$ it would be $w_B=-P_2 \Delta V$.  For each step $\Delta U=U_f-U_i=5/2 N k_b (T_f-T_i)$.  Note that we are dealing with a ideal gas thus $PV=Nk_bT$, and so $\Delta U=5/2(P_fV_f-P_iV_i)$.  Now computing the changes in internal energy we get $\Delta U_A=5/2V_1\Delta P$, $\Delta U_B=5/2P_2\Delta V$, $\Delta U_C=-5/2V_2\Delta P$, $\Delta U_D=-5/2P_1\Delta V$.  Noting that $Q=\delta U-w$ we see $Q_A=5/2V_1\Delta P$, $Q_B=5/2P_2\Delta V+P_2 \Delta V=7/2P_2\Delta V$, $Q_C=-5/2V_2\Delta P$, $Q_D=-7/2P_1\Delta V$.
\item
In $B$ a fixed force is applied to the piston while the gas is heated, allowing the gas to expand under constant pressure.  In $C$ the piston is locked and the gas is cooled. In $D$ the gas is heated while a constant force is applied to the piston.
\item
The net work would be $w_B+w_D=P_1 \Delta V-P_2 \Delta V=-\Delta P \Delta V$.  The net change in internal energy would be zero since this is a cycle and internal energy is a state function, to confirm $\Delta U_{total}=5/2(V_1\Delta P+P_2\Delta V-V_2\Delta P-P_1\Delta V)=5/2(-\Delta V\Delta P+\Delta P\Delta V)=0$.  The heat added to the system should be exactly the negation of the work done on the system thus $Q_{total}=\Delta P \Delta V$.  In conclusion this is a cycle where net heat is added and work is extracted.
\end{enumerate}
\end{exercise}



\end{document}